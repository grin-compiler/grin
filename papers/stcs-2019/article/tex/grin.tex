\documentclass[main.tex]{subfiles}
\begin{document}
	
	GRIN is short for \emph{Graph Reduction Intermediate Notation}. GRIN consists of an intermediate representation language (IR in the followings) as well as the entire compiler back end framework built around it. GRIN tries to resolve the issues highlighted in Section~\ref{sec:intro} by using interprocedural whole program optimization. 
	
	Interprocedural program analysis is a type of data-flow analysis that propagates information about certain program elements through function calls. Using interprocedural analyses instead of intraprocedural ones, allows for optimizations across functions. This means the framework can handle the issue of large sets of small interconnecting functions presented by the composable programming style. 
	
	Whole program analysis enables optimizations across modules. This type of data-flow analysis has all the available information about the program at once. As a consequence, it is possible to analyze and optimize global functions. Furthermore, with the help of whole program analysis, laziness can be made explicit. In fact, the evaluation of suspended computations in GRIN is done by an ordinary function called \pilcode{eval}. This is a global function uniquely generated for each program, meaning it can be optimized just like any other function by using whole program analysis. 
	
	Finally, since the analyses and optimizations are implemented on a general intermediate representation, all other languages can benefit from the features provided by the GRIN back end. The intermediate layer of GRIN between the front end language and the low level machine code serves the purpose of eliminating functional artifacts from programs. This is achieved by using optimizing program transformations specific to the GRIN IR and functional languages in general. The simplified programs can then be optimized further by using conventional techniques already available. For example, it is possible to compile GRIN to LLVM and take advantage of an entire compiler framework providing a huge array of very powerful tools and features.
	
	% TODO: refer LLVM section
	
\end{document}